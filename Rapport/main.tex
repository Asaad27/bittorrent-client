\documentclass{article}
\usepackage[utf8]{inputenc}
\usepackage[export]{adjustbox}

\RequirePackage[french]{babel} %Langue du document
\RequirePackage[utf8]{inputenc} %Caractères spéciaux
\RequirePackage[section]{placeins}%Pour placement de section
\RequirePackage[T1]{fontenc} %Quelques lettres qui sont pas inclus dans UTF-8
\RequirePackage{mathtools} %Paquet pour des équations et symboles mathématiques
\RequirePackage{siunitx} %Pour écrire avec la notation scientifique (Ex.: \num{2e+9})
\RequirePackage{float} %Pour placement d'images
\RequirePackage{graphicx} %Paquet pour insérer des images
\RequirePackage[justification=centering]{caption} %Pour les légendes centralisées
\RequirePackage{subcaption}
\RequirePackage{wallpaper}
\RequirePackage{nomencl}
%\makenomenclature
\RequirePackage{fancyhdr}
%\pagestyle{fancy}
%\fancyheadoffset{1cm}
%\setlength{\headheight}{2cm}
\RequirePackage{url}
\RequirePackage[hidelinks]{hyperref}%Paquet pour insérer légendes dans des sous-figures comme Figure 1a, 1b
\RequirePackage[left=3cm,right=3cm,top=3cm,bottom=5cm]{geometry} %Configuration de la page


\begin{document}
\begin{titlepage}

\centering
\begin{figure}
   \includegraphics[width=0.4\textwidth]{logophelma.jpg}
   \hfill
   \includegraphics[width=0.4\textwidth]{logoimag.png}
\end{figure}

\rule{\linewidth}{0.2 mm} \\[0.4 cm]
{\huge\bfseries Rapport de Projet Réseaux \par\vspace{1cm}}
{\Large Réalisation d'un Client Bittorent en Java}
\rule{\linewidth}{0.2 mm} \\[1.0 cm]



{\scshape\Large
Équipe 5 : Youssef Boulkhir, Asaad Belarbi, Pierre Cheylus
\par}
\vspace{1cm}

{\scshape \Large Phelma 3A 2021-2022 \\
Filière Systèmes Embarqués et Objets Connectés \par}
\vspace{7cm}

\begin{flushleft}
\emph{\textbf{Superviseur école :}}\\
\textsc{Olivier Alphand} \\
olivier.alphand@univ-grenoble-alpes.fr\\
\end{flushleft}
\end{titlepage}
\newpage

\tableofcontents
\newpage

\section{Introduction}
% Nécessaire ?

\section{Implémentation}
    \subsection{Tableau des fonctionnalités}
    \subsection{Architecture}
    % Ici on pourrait mettre le diagramme UML et parler des design patterns utilisés
    \subsection{Traitement de messages}
    \subsection{Algorithme de sélection des pièces}

\section{Performances}
    \subsection{Scénario de Test}
    \subsection{Tableau de performances}

\section{Tests \& Validation}

\section{Organisation du travail}
    \subsection{Organisation générale}
    \subsection{Conclusions personnelles}


\end{document}
